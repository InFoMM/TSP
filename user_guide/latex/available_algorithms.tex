\section{Available algorithms}
\label{sec:Available_algorithms}

A list of various algorithms and a brief description is provided in Table~\ref{tab:brief_algorithm_descriptions}. For more comprehensive descriptions see Section~\ref{sec:algorithm_descriptions}.

\begin{table}[hbt]
\begin{center}
% Try to input a short description (no more than 1 or 2 lines) of algorithms in alphabetical order.
\begin{tabularx}{\textwidth}{|L{0.4\textwidth}|X|}
\hline 
\multicolumn{1}{|c|}{\textbf{Algorithm}}  % use \texttt{text} instead of \verb|text| to allow line breaks for long function names.
& \multicolumn{1}{c|}{\textbf{Description}} \\
\hline 
\texttt{forcefully\_increasing\_loop.m} & Begins with a small loop, progressively increasing the size of the route by forcefully including a random city by changing the route by the amount. \\
\hdashline
\texttt{increasing\_loop.m} & Begins with a small loop, progressively increasing the size of the route by including the city which increases the distance by the minimal amount. \\
\hdashline
\texttt{tsp\_ip\_cut\_set\_oliver.m} &  Integer programming solver which performs the cut-set constraint until no sub-loops exist. \\
\hdashline
\texttt{tsp\_ip\_no\_cut\_set\_oliver.m} & Integer programming solver which allows sub-loops.\\
\hdashline
\texttt{tsp\_lp\_no\_cut\_set\_oliver.m} & Linear programming solver which allows sub loops and partial journeys (non-physical).\\
\hline
\end{tabularx}
\caption{Available algorithms for solving the travelling salesman problem, giving the function name and a brief description.}
\label{tab:brief_algorithm_descriptions}
\end{center}
\end{table}

\clearpage