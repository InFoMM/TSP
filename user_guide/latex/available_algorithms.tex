\section{Available algorithms}
\label{sec:Available_algorithms}

A list of various algorithms and a brief description is provided in Table~\ref{tab:brief_algorithm_descriptions}. For more comprehensive descriptions see Section~\ref{sec:algorithm_descriptions}.

\begin{table}[hbt]
\begin{center}
\footnotesize
% Try to input a short description (no more than 1 or 2 lines) of algorithms in alphabetical order.
\begin{tabularx}{\textwidth}{|L{0.3\textwidth}|X|}
\hline 
\multicolumn{1}{|c|}{Algorithm}  % use \texttt{text} instead of \verb|text| to allow line breaks for long function names.
& \multicolumn{1}{c|}{Description} \\
\hline 
\texttt{edgeupperbound.m} & Adds the cheapest unused edge that won't create a fork or a loop until a path is formed, then adds in the final edge to create a loop. \\
\hdashline
\texttt{FD\_bruteForce.m} & Explores all possible paths recursively tracking the total distance of smaller networks. \\
\hdashline
\texttt{FD\_dynamicProgramming.m} & Explores all possible paths ending in a given city, iteratively increasing the network size using the Held-Karp algorithm. \\
\hdashline
\texttt{FD\_greedy.m} & Greedy algorithm increases the path size by including the nearest city. \\
\hdashline
\texttt{FD\_LPTSP.m} &  Integer linear programming solver. \\
\hdashline
\texttt{FD\_LPTSPit.m} &  Integer linear programming solver which ignores the cut-set constraint, but does constrain against sub-cycles. \\
\hdashline
\texttt{FD\_stochastic.m} & Random permutations of paths are considered for a given number of trials. \\
\hdashline
\texttt{forcefully\_increasing\_loop.m} & Begins with a small loop, progressively increasing the size of the route by forcefully including a random city by changing the route by the amount. \\
\hdashline
\texttt{greedy\_algorith\_TSP.m} & Greedy algorithm moves to shortest edge, not allowing for sub-loops. \\
\hdashline
\texttt{greedy\_algorith\_TSP\_all.m} & Greedy algorithm moves to shortest edge, not allowing for sub-loops, and uses a specific starting point. \\
\hdashline
\texttt{increasing\_loop.m} & Begins with a small loop, progressively increasing the size of the route by including the city which increases the distance by the minimal amount. \\
\hdashline
\texttt{IntLinProgCutSetTSP.m} & Integer programming method without the cut-set constraint. \\
\hdashline
\texttt{linprogtsp2.m} &  Miller-Tucker-Zemlin  algorithm, uses cut-set constraints. \\
\hdashline
\texttt{nearestneigh.m} &  A greedy nearest neighbours algorithm added the shortest possible edge.  \\
\hdashline
\texttt{optimal\_greedy\_TSP.m} &  A greedy nearest neighbours algorithm using a specific starting point.  \\
\hdashline
\texttt{RG\_stochastic.m} &  Generates random paths iteratively saving the best result for a chosen number of iterations. \\
\hdashline
\texttt{search\_permutations.m} &  Exhaustive search over all path permutations. \\
\hdashline
\texttt{stochastic\_TSP.m} &  Randomly generates paths, using the shortest over a fixed number of iterations. \\
\hdashline
\texttt{tabu\_search.m} &  Performs a Tabu search. \\
\hdashline
\texttt{tsp\_ip\_no\_cut\_set\_oliver.m} & Integer programming solver which allows sub-loops.\\
\hdashline
\texttt{tsp\_lp\_no\_cut\_set\_oliver.m} & Linear programming solver which allows sub loops and partial journeys (non-physical).\\
\texttt{two\_opt\_search.m} &  Performs a 2-opt local search. \\
\hdashline
\texttt{TwoHeadedSnake.m} &  A two headed greedy algorithm for finding a Hamiltonian cycle by adding the nearest two cities iteratively. \\
\hdashline
\texttt{twoopt.m} &  Performs a 2-opt local search. \\
\hline
\end{tabularx}
\caption{Available algorithms for solving the travelling salesman problem, giving the function name and a brief description.}
\label{tab:brief_algorithm_descriptions}
\end{center}
\end{table}

\clearpage