\section{Instructions}
\label{sec:intructions}


\subsection{Getting started}
\label{subsec:getting_started}

\subsubsection{Pulling from GitHub}
\label{subsubsec:pulling_from_github}

The code for this suite can be found at 
\begin{center}
\href{https://github.com/InFoMM/TSP}{https://github.com/InFoMM/TSP}
\end{center}
which should be cloned/forked into whichever directory the user wants to call the functions from.

\subsubsection{Git LFS}
\label{subsubsec:git_lfs}

To fully pull some of the larger files this Git repository, it will be necessary to have Git LFS (Large File System) installed. To check if this is installed simply type into the console \\
\indent
\verb|$ git lfs| \\
If this is not installed, don't worry, installing it is easy, and  instructions can be found at 
\begin{center}
	\href{https://git-lfs.github.com/}{https://git-lfs.github.com/}.
\end{center}


\subsubsection{Git branches}
\label{subsubsec:git_branches}

To see which branches are available type \\
\indent
\verb|$ git branch| \\
into the console. To switch to one of these branches, type \\
\indent
\verb|$ git checkout <available branch name>| \\
This is because many of the authors have to present this work, and so ensures submitted work is separate from the \texttt{master} branch.

\subsection{Directory structure}
\label{subsec:directory_structure}

Once the directory has been cloned it should have the following directory structure:
\\
\noindent
\begin{minipage}{\linewidth}
\begin{verbatim}
    TSP/
        README.txt
        matlab/
            increasing_loop.m
            performance.m
            etc ...
        user_guide/
            figures/
            latex/
                guide.pdf
                etc ...
\end{verbatim}
\end{minipage}





\subsubsection{Working paths}
\label{subsubsec:working_paths}

To use the functions in \texttt{TSP/matlab} change the Matlab current working directory into this folder. \textbf{Do not add this folder or sub-folders to the working path!} This is because many of these functions use relative path location descriptions to find supporting functions. 

\subsection{Dependencies}
\label{subsec:dependencies}

Unfortunately not all the dependencies are known, but on MATLABR2016b these likely include:
\begin{itemize}
	\item Optimisation Toolbox.
	\item Statistics and Machine Learning.
\end{itemize}
If any others are found please let us know via GitHub.


\subsection{Testing it works}
\label{subsec:testing_it_works}

The directory \texttt{Matlab/} contains the file \verb|test.m| which when run should output the following: \\
\noindent
\begin{verbatim}
Performance of:
        Purmuation searching algorithm.
Time taken:
    2.686500e-01
Total distance:
    2999
Paths taken:
    [1, 1, 10, 3, 2, 5, 6, 9, 8, 7]
Paths taken:
    [4, 4, 7, 8, 9, 6, 5, 2, 3, 10]
Routes taken:
    [Alicante, Alicante, Valencia, Granada, Barcelona, ...
     Malaga, Pamplona, Sevilla, Santander, Salamanca]
Routes taken:
    [Madrid, Madrid, Salamanca, Santander, Sevilla, ...
    Pamplona, Malaga, Barcelona, Granada, Valencia]
\end{verbatim}
where the time taken to complete may be different to the quoted value. 

\subsubsection{Trying various algorithms}
\label{subsubsec:trying_various_algorithms}

To see the performance of some different algorithms see \verb|performance.m|, where we have presented some example code to demonstrate some performance characteristics of various algorithms.

\clearpage