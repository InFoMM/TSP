\subsection{\texttt{tsp\_ip\_cut\_set\_oliver}}
\label{subsec:tsp_ip_cut_set_oliver}

\begin{flushright}
\textbf{Author} \\
Oliver Sheridan-Methven
\end{flushright}

This formulates the travelling salesman problem into an integer programming problem, and then iteratively performs a version of the cut set constraint  to ensure no sub-loops are allowed.

We can represent the edges that connect any two cities $ i $ and $ j $ by a $ n \times n $ matrix $ x_{ij} $, taking value on $ \{0, 1\} $, which represent an edge being absent or present respectively. We impose the constraint that there is only one path in and out of a city by $ \sum_{i} x_{ij} = 1 \: \forall j $ and  $ \sum_{j} x_{ij} = 1 \: \forall i $. We then input this formulation into the Matlab \verb|intlinprog| solver where we request integer solutions.

The output produced does possibly contain loops, so to prevent this after the first iteration we form a constraint constructed as follows. Suppose $ m $ sub-loops are formed, where for the $ p_1 $-th sub-loop has indices in the set $ \mathbb{S}_{p_1} $. We know that a way to avoid this sub-loop is to force a revised solution to contain at least one edge from any index in $ \mathbb{S}_{p_1} $ to any index in $ \mathbb{S}_{p_1}^c $. Hence the $ p_1 $-th loop gives us one inequality constraint. As this holds for each of the ${p_1} $ sub-loops we have ${p_1} $ constraints after this iteration.

Using our $ {p_1} $ constraints we repeat the integer solver, which is guaranteed then to not contain any of the prior $ {p_1} $ sub-loop. This new solution might then contain ${p_2} $ sub-loops. We can extend our constraint condition to now include $ p_1 + p_2 $ constraints. We iterate over the following process, where our constraint matrix increases on each iteration until there are no sub-loops, giving the an optimal exact solution (not necessarily unique).

\subsubsection{\texttt{tsp\_ip\_no\_cut\_set\_oliver}}
\label{subsubsec:tsp_ip_no_cut_set_oliver}

\begin{flushright}
\textbf{Author} \\
Oliver Sheridan-Methven
\end{flushright}

Identical to the \verb|tsp_ip_cut_set_oliver| in Section~\ref{subsec:tsp_ip_cut_set_oliver} except we do not impose the cut set constraint. The solutions produced allow sub-loops.
\subsubsection{\texttt{tsp\_lp\_no\_cut\_set\_oliver}}
\label{subsubsec:tsp_lp_no_cut_set_oliver}

\begin{flushright}
\textbf{Author} \\
Oliver Sheridan-Methven
\end{flushright}

Identical to the \verb|tsp_ip_cut_set_oliver| in Section~\ref{subsec:tsp_ip_cut_set_oliver} except we do not impose the cut set constraint and use a linear programming solver rather than an integer programming solver\footnote{We use Matlab's \texttt{linprog}.}. 

The solutions produced allow for non-integer connections, which gives an unphysical solution to the problem. Furthermore, the nearest integer to the solution may not be feasible. This can be seen if we consider the element of $ x_{ij} $ which is non-integer, e.g. $ \frac{1}{2} $. if we force one of these values of $ \frac{1}{2} \to 1$ and the other to zero, then there is no guarantee that the resulting path may not respect the entering and leaving a city only once constraint\footnote{This can be seen more formally by noting that the entering and leaving a city only once constraint corresponds to $ x_{ij} $ having to be a latin-hypercube.}.

