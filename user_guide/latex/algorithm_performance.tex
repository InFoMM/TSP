\section{Algorithm performance}
\label{sec:algorithm_performance}

We have two classes of algorithms, those which compute exact solutions, and those which compute approximate solutions. Those which compute exact solutions are accurate but may take a long time to execute, whereas the approximate solutions may be quick to run, but may give routes which may not be optimal (although possibly close to optimal). With any algorithm a compromise between execution time and accuracy has to be met, and so  depending on the need of the user, we present some characteristics of the algorithms available.

\subsection{Execution times}
\label{subsec:execution_times}

We benchmark the average execution time of algorithms on the Spanish data set from Table~\ref{tab:distance_matrix} over 100 trials. Furthermore, generating random symmetric distance matrices of various sizes, we can see how the algorithms scale with the number of cities. The results are summarised  in Table~\ref{tab:execution_times}.

\begin{table}[htb]
	\footnotesize
\begin{center}
\begin{tabular}{|L{0.3\textwidth}|c@{\hspace{1ex}}|c@{\hspace{1ex}}c@{\hspace{1ex}}c@{\hspace{1ex}}c@{\hspace{1ex}}c@{\hspace{1ex}}c@{\hspace{1ex}}c@{\hspace{1ex}}|}
\hline
\multicolumn{1}{|c|}{Algorithm}          & Spain & 6 & 8 & 10 & 12 & 15 & 20 & 30 \\ \hline
\texttt{increasing\_loop}                &   $ 1.6 \! \cdot \! 10^{-4} $   & $ 2.2  \! \cdot \! 10^{-4} $ & $ 1.0  \! \cdot \! 10^{-4} $ & $ 1.0  \! \cdot \! 10^{-4} $ & $ 1.4 \! \cdot \!10^{-4} $ & $ 1.8 \! \cdot \! 10^{-4} $ & $ 2.7 \! \cdot \! 10^{-4} $ & $ 8.5 \! \cdot \! 10^{-4} $ \\
\texttt{forcefully\_increasing\_loop}    &  $ 3.1 \! \cdot \! 10^{-4}$  & $ 6.7 \! \cdot \! 10^{-5}$ &  $ 9.6 \! \cdot \! 10^{-5}$ &  $ 9.1 \! \cdot \! 10^{-5}$ &  $ 1.3 \! \cdot \! 10^{-4}$ &  $ 1.7 \! \cdot \! 10^{-4}$ &  $ 7.2 \! \cdot \! 10^{-4}$ &  $ 4.6 \! \cdot \! 10^{-4}$ \\
\texttt{tsp\_ip\_cut\_set\_oliver.m}     &    $ 5.7 \! \cdot \! 10^{-1}$   &  $ 9.1 \! \cdot \! 10^{-2}$ &  $ 2.0 \! \cdot \! 10^{-1}$ &  $ 1.8 \! \cdot \! 10^{0}$ &  \textasteriskcentered &\textasteriskcentered & \textasteriskcentered & \textasteriskcentered \\
\texttt{tsp\_ip\_no\_cut\_set\_oliver.m} &   s   & 6 & 8 & 10 & 12 & 15 & 20 & 30 \\
\texttt{tsp\_lp\_no\_cut\_set\_oliver.m} &   s   & 6 & 8 & 10 & 12 & 15 & 20 & 30 \\ \hline
\end{tabular}
\end{center}
\caption{text}
\label{tab:execution_times}
\end{table}

\subsection{Comparison to optimality}
\label{subsec:comparison_to_optimality}

To see how accurate our algorithms are, we need to know what the optimal solution is. We present in Table~\ref{tab:optimality} the average fractional discrepancy error from optimality for the Spanish cities example from Table~\ref{tab:distance_matrix}.

\begin{table}[htb]
\begin{center}
\begin{tabular}{|L{0.4\textwidth}|c|}
\hline
\multicolumn{1}{|c|}{Algorithm}          & Error \\ \hline
\texttt{increasing\_loop}                &   5.7   \\
\texttt{forcefully\_increasing\_loop}    &   3.2   \\
\texttt{tsp\_ip\_cut\_set\_oliver.m}     &   -   \\
\texttt{tsp\_ip\_no\_cut\_set\_oliver.m} &   s   \\
\texttt{tsp\_lp\_no\_cut\_set\_oliver.m} &   s   \\ \hline
\end{tabular}
\end{center}
\caption{text}
\label{tab:optimality}
\end{table}

\clearpage